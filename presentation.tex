\documentclass{beamer}

\usepackage[utf8]{inputenc}
\usepackage[english]{babel}

\usepackage{graphicx}
\usepackage{amsmath}
\usepackage{amssymb}
\usepackage{amsthm}
\usepackage{array}

\usepackage{alltt}

% Multiple columns
\usepackage{multicol}

\addtocounter{footnote}{1}
\setcounter{tocdepth}{5}
\setcounter{secnumdepth}{5}
\renewcommand{\floatpagefraction}{0.75}

%\usetheme{CambridgeUS}
\usetheme{Warsaw}

%Information to be included in the title page:
\title{Modularized C++}
\subtitle{A modern approach to C++ project setup}
\author{Alexander Christensen}
\institute{}
\date[VLC]
{Copenhagen C/C++ Meetup, August 2022}

\beamertemplatenavigationsymbolsempty

%% Reference an equation, a figure, or a section

%% \secref{label} - make a reference to a section
\newcommand{\secref}[1]{Section~\ref{#1}}

%% \eqref{reference} - make a reference to an equation
%%\newcommand{\eqref}[1]{(\ref{#1})}

%% \figref{reference} - make a reference to an figure
\newcommand{\figref}[1]{Figure~\ref{#1}}

\newcommand{\basetop}[1]{\vtop{\vskip-1ex\hbox{#1}}}
\newcommand{\source}[1]{\let\thefootnote\relax\footnotetext{\scriptsize\textcolor{kugray1}{Source: #1}}}

%\bibliographystyle{longalpha}
%\bibliography{refs}

%% -*- Mode: latex -*-

%% Macros defined during a long time and used much
% plus - a plus sign
\newcommand{\plus}{+}

% minus - a minus having the same width as a plus
\newlength{\minuswidth}
\settowidth{\minuswidth}{+}
\newlength{\minusheight}
\settoheight{\minusheight}{+}
\newcommand{\minus}{\rule[0.5\minusheight]{\minuswidth}{0.5pt}}

% The basis vector standard
%\renewcommand{\vec}[1]{\boldsymbol{#1}}
\newcommand{\grad}{\operatorname{\nabla}}
\newcommand{\curl}{\operatorname{\text{curl}}}
\newcommand{\divergence}{\operatorname{\text{div}}}
\newcommand{\vecop}{\operatorname{\text{vec}}}
\newcommand{\diag}{\operatorname{\text{diag}}}
\renewcommand{\Re}{\mathbb{R}}
\newcommand{\Co}{\mathbb{C}}
\newcommand{\In}{\mathbb{Z}}
\newcommand{\sign}{\operatorname{sgn}}
%\newcommand{\trace}{\operatorname{Tr}}
\newcommand{\arctantwo}{\ensuremath{\arctan\!2}}
%\newcommand{\mat}[1]{\ensuremath{\boldsymbol{#1} }}
\newcommand{\I}{\mat{1}}
\newcommand{\crossmat}[1]{\ensuremath{\boldsymbol{#1}^{\times} }}
\newcommand{\jacobian}[1]{\ensuremath{\boldsymbol{\mathit{#1}} }}
\newcommand{\set}[1]{\ensuremath{ \boldsymbol{#1} }}
\newcommand{\func}[1]{{\bf{#1}}}
\newcommand{\enorm}[1]{\ensuremath{\left\| #1 \right\|_{_2}}}
%\newcommand{\norm}[1]{\ensuremath{\left\| #1 \right\|}}
\newcommand{\bdet}[1]{\ensuremath{\left| #1 \right|}}
\newcommand{\abs}[1]{\ensuremath{\left| #1 \right|}}
\newcommand{\rtm}{$^{\textrm{®}}$}


\newcommand{\kenny}[1]{ #1 }
\newcommand{\henrik}[1]{ #1 }


%% \maya - the maya signature
\newcommand{\maya}{$ \texttt{Maya}^{\text{\texttrademark}} $}

%% \fat{symbol} - make this symbol fat
\newcommand{\fat}[1]{\mathit{\mathbf{#1}}}
%% \newcommand{\fat}[1]{\hbox{\boldmath $ #1 $}}

%% \vec{symbol} - Vector
%% \renewcommand{\vec}[1]{\mathbf{#1}}
\renewcommand{\vec}[1]{\fat{#1}}

%% \mat{symbol} - Matrix
%%\newcommand{\mat}[1]{\ensuremath{\boldsymbol{#1} }}
\newcommand{\mat}[1]{\fat{#1}}

%% \ezero,...,\ethree - the spin matrices
\newcommand{\ezero}{\begin{bmatrix} 1 & 0 \\  0 &  1 \end{bmatrix}}
\newcommand{\eone}{\begin{bmatrix} i & 0 \\  0 & -i \end{bmatrix}}
\newcommand{\etwo}{\begin{bmatrix} 0 & 1 \\ -1 &  0 \end{bmatrix}}
\newcommand{\ethree}{\begin{bmatrix} 0 & i \\  i &  0 \end{bmatrix}}

%% \quat{symbol} - Quaternion
%% \newcommand{\quat}[1]{\mathbf{#1}}
\newcommand{\quat}[1]{\fat{#1}}

%% \real{quaternion} - the real (scalar) part of a quaternion
\newcommand{\real}{\operatorname{real}}

%% \pure{quaternion} - the pure (vector) part of a quaternion
\newcommand{\pure}{\operatorname{pure}}

%% \sgn{symbol} - the sign of a symbol
\newcommand{\sgn}{\operatorname{sgn}}

%% \norm{symbol} - Norm of a vector/quaternion
\newcommand{\norm}[1]{\parallel {#1} \parallel}

%% \trace{matrix}
\newcommand{\trace}[1]{\mathrm{trace}(#1)}

% This is for code-snippets in the text.
\usepackage{fancyvrb}   %% Try to comment this out if problems with pdflatex
\newcommand{\code}[1]%
%{ \VerbatimInput[frame=none,fontsize=\footnotesize,numbers=none,label=\texttt{#1}]{#1}  }
{ \VerbatimInput[frame=single,fontsize=\footnotesize,numbers=left,label=\texttt{#1}]{#1}  }


%\newcommand{\todo}[1]{ {\Bf Todo:} #1}
\newcommand{\todo}[1]{ }

%\newcommand{\longversion}[1]{ #1 }
\newcommand{\longversion}[1]{ }


%% Kennys pseudocode environment

\newenvironment{pseudocode}{
  \begin{center}
    \begin{minipage}[t]{0.8\columnwidth}
      \footnotesize
      \rule{\columnwidth}{1pt}
    }{
      \rule{\columnwidth}{1pt}
    \end{minipage}
  \end{center}
}

{
\AtBeginSection[wef]
{
\begin{frame}
\frametitle{Table of Contents}
\tableofcontents[currentsection]{1}
\end{frame}
}
}



\begin{document}


%
% FRONT PAGE
%
\begin{frame}[plain]
    \titlepage
\end{frame}


%
% TABLE OF CONTENTS
%
\begin{frame}[plain]
\frametitle{Contents}
\tableofcontents
\end{frame}


%
% MOTIVATION
%
\begin{frame}[plain]
\frametitle{Motivation}
\section{Motivation}
\end{frame}


%
% EVOLUTION OF C++
%
\begin{frame}[plain]
\frametitle{Evolution of C++}
\section{Evolution of C++}
\end{frame}


%
% STATUS QUO
%
\begin{frame}[plain]
\frametitle{Status Quo}
\section{Status Quo}
\end{frame}


%
% HEADER FILES ARE PROBLEMATIC
%
\begin{frame}[plain]
\frametitle{Header files are problematic}
\section{Header files}
\end{frame}


%
% BLANK SLATE IS IMPOSSIBLE
%
\begin{frame}[plain]
\frametitle{Blank slate is impossible}
\section{Blank slate?}
\end{frame}


%
% THE MODULES PROPOSAL
%
\begin{frame}[plain]
\frametitle{The modules proposal}
\section{The proposal}
\end{frame}


%
% WHAT IS A MODULE
%
\begin{frame}[plain]
\frametitle{What is a module?}
\section{What is a module?}
\end{frame}


%
% THE PROMISE OF MODULES
%
\begin{frame}[plain]
\frametitle{The promise of modules}
\section{The promise of modules}
\begin{itemize}
\item That we may rid our projects of header files
\item No more include directories
\item No more inline functions and methods everywhere
\item No more "header-only" libraries
\item No more magic macro customizations for building our code
\item The preprocessor has no understanding of types, but the compiler
has - let's use it!
\item Everything will be easier, simpler, and better
\item My estimate: $\sim$50\% language complexity reduction
\item Vastly improved build times
\item No loss of expressiveness
\end{itemize}
\end{frame}


%
% THE BUILDING BLOCKS
%
\begin{frame}[plain]
\frametitle{The building blocks}
\section{The building blocks}
We get 4 new "classifications" of files:
\begin{itemize}
\item Header unit (temporary solution of creating a BMI from a header file)
\begin{itemize}
\item \texttt{g++ -std=c++20 -fmodules-ts -xc++-system-header iostream}
\item Creates BMI in \texttt{./gcm.cache/usr/include/c++/11/iostream.gcm}
\end{itemize}
\item Module interface unit / primary module interface unit
\begin{itemize}
\item This is a translation unit which exports a module
\end{itemize}
\item Module partition / submodule
\begin{itemize}
\item Another translation unit which belongs to a module interface
\end{itemize}
\item Module implementation unit
\begin{itemize}
\item A translation unit which may provide implementations to declarations
in module interface
\end{itemize}
\end{itemize}
\end{frame}


%
% MODULE INTERFACE FILE STRUCTURE
%
\begin{frame}[plain]
\frametitle{Module interface file structure}
\section{File structure}
\begin{multicols}{2}
\begin{alltt}\scriptsize
module;\\

\#define NDEBUG\\
\#include <assert.h>\\

export module foo;\\
\vspace{7mm}

import <string>;\\
\vspace{6.5mm}

export \{\\
\qquad int magic\_value() { return 42; }\\
\}
\columnbreak

$\leftarrow$ global module fragment (\textit{May be used for preprocessor directives. Not required.})\\
\vspace{3.5mm}

$\leftarrow$ module declaration (\textit{The rest of the file is considered part of this module.})\\

$\leftarrow$ import declaration (\textit{Other modules or header units may be imported.})\\

$\leftarrow$ export declaration (\textit{Everything inside is visible for consumers of the module.})

\end{alltt}
\end{multicols}
\end{frame}



\end{document}
